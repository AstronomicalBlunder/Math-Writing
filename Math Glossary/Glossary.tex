\documentclass[12pt,x11names]{article}
\setlength{\headheight}{22pt}

% Typesetting Packages
\usepackage[x11names]{xcolor}

\usepackage[utf8]{inputenc}
\usepackage[english]{babel}
\usepackage[a4paper, total={7in, 9.5in}]{geometry}
\usepackage{setspace}

\usepackage{import}
\usepackage{csquotes}
\usepackage[style=alphabetic, backend=bibtex8,
            autocite=footnote, notetype=endonly, labeldateparts]{biblatex}
\usepackage{imakeidx}

%%%%%%%%%%%%%%% HEADERS/FOOTERS %%%%%%%%%%%%%%%%%%%%
\usepackage{fancyhdr}
\pagestyle{fancy}
\fancyhead[L]{How to Read Math}
\rhead{\footnotesize Questions and Suggestions to:\\ aaryan11@stanford.edu}
%%%%%%%%%%%%%%%%%%%%%%%%%%%%%%%%%%%%%%%%%%%%%%%%%%%%%


%%%%%%%%%%%%%%%%%%% COUNTER(S)  %%%%%%%%%%%%%%%%%%%%
\newcounter{exercises}
\counterwithin{exercises}{subsection}
\renewcommand{\theexercises}{\thesubsection.\Alph{exercises}}
%%%%%%%%%%%%%%%%%%%%%%%%%%%%%%%%%%%%%%%%%%%%%%%%%%%%%

% Math Packages
\usepackage{amsthm, amssymb, amsmath, centernot, graphicx}
\usepackage{thmtools}
\usepackage{stmaryrd}
\usepackage[shortlabels]{enumitem}
\setlistdepth{9}
% Fonts

\usepackage{calligra, mathrsfs}
\usepackage{lmodern}
\usepackage[T2A,T1]{fontenc}

% Other Font Options
% \usepackage{mathpazo} % The Rising Sea
% \usepackage{euler}

% \usepackage{euler} % Vaughan's notes
% \usepackage{eulervm}

%\usepackage{kpfonts}  

\usepackage{libertine} 
\usepackage{libertinust1math}

%\usepackage{amsfonts} % Stacks project math font


% Graphics Packages
\usepackage{tikz-cd}
\usepackage{graphicx}
\usepackage{framed}
\usepackage[many]{tcolorbox}
\usepackage[framemethod=TikZ]{mdframed}
\usepackage{pagecolor}
\usepackage{etoolbox}
\usepackage{listings}
\definecolor{shadecolor}{gray}{0.95}


% General Utilities 

\newcommand{\ul}[1]{\underline{#1}}
\newcommand{\ol}[1]{\overline{#1}}
\newcommand{\bbar}[1]{\overline{#1}}
\renewcommand{\bf}[1]{\mathbf{#1}}
\renewcommand{\labelenumi}{(\alph{enumi})}
\usepackage{hyperref}
\usepackage{lipsum}
\newcommand{\chref}[2]{\href{#1}{\color{blue}{\underline{#2}}}}
\usepackage{cleveref}



%%%%%%%%%%%%%%%%%%%%%%%%%%%%%%%%%%%%%%%%%%%%%%%%%%%%%%%%%%%%
%%%%%%%%%%%%%%%%%%%% CUSTOM COMMANDS %%%%%%%%%%%%%%%%%%%%%%%
%%%%%%%%%%%%%%%%%%%%%%%%%%%%%%%%%%%%%%%%%%%%%%%%%%%%%%%%%%%%

%%%%%%%%%%%%%% STYLIZED LETTERS %%%%%%%%%%%%%%%%
%% BOLD LETTERS %%
\newcommand{\bx}{\mathbf{x}}
\newcommand{\by}{\mathbf{y}}
\newcommand{\bv}{\mathbf{v}}
\newcommand{\bu}{\mathbf{u}}
\newcommand{\bw}{\mathbf{w}}
\newcommand{\zero}{\mathbf{0}}
\newcommand{\one}{\mathbb{1}}
\newcommand{\nn}{\mathbb{N}}
\newcommand{\zz}{\mathbb{Z}}
\newcommand{\qq}{\mathbb{Q}}
\newcommand{\rr}{\mathbb{R}}
\newcommand{\cc}{\mathbb{C}}
\newcommand{\ff}{\mathbb{F}}
\newcommand{\rp}{\mathbb{RP}}
\newcommand{\cp}{\mathbb{CP}}
\renewcommand\tt{\mathbb{T}}
\newcommand{\bp}{\mathbb{P}}
\newcommand{\boldp}{\mathbb{P}}
\renewcommand\aa{\mathbb{A}}
\newcommand\ii{\mathbb{I}}
\newcommand{\ee}{\mathbb{E}}
% MATHFRAK LETTERS %
\newcommand{\pp}{\mathfrak{p}}
\newcommand{\PP}{\mathfrak{P}}
\newcommand{\frakq}{\mathfrak{q}}
\newcommand{\frakQ}{\mathfrak{Q}}
\newcommand{\mm}{\mathfrak{m}}
\newcommand{\MM}{\mathfrak{M}}
\newcommand{\frako}{\mathfrak{o}}
\newcommand{\fraka}{\mathfrak{a}}
\newcommand{\frakb}{\mathfrak{b}}
\newcommand{\frakc}{\mathfrak{c}}
\newcommand{\frakd}{\mathfrak{d}}
\newcommand{\frakD}{\mathfrak{D}}
\newcommand{\frakn}{\mathfrak{n}}
\newcommand{\frakf}{\mathfrak{f}}
\newcommand{\frakg}{\mathfrak{g}}
\newcommand{\jj}{\mathfrak{j}}
%% MATHCAL LETTERS %%
\newcommand{\calc}{\mathcal{C}}
\newcommand{\cald}{\mathcal{D}}
\newcommand{\calf}{\mathcal{F}}
\newcommand{\calg}{\mathcal{G}}
\newcommand{\calh}{\mathcal{H}}
\newcommand{\cala}{\mathcal{A}}
\newcommand{\calb}{\mathcal{B}}
\newcommand{\cale}{\mathcal{E}}
\newcommand{\calo}{\mathcal{O}}
\newcommand{\cali}{\mathcal{I}}
\newcommand{\calj}{\mathcal{J}}
\newcommand{\call}{\mathcal{L}}
\newcommand{\calx}{\mathcal{X}}
\newcommand{\caly}{\mathcal{Y}}
\newcommand{\calz}{\mathcal{Z}}
\newcommand{\calm}{\mathcal{M}}
\newcommand{\caln}{\mathcal{N}}
\newcommand{\cals}{\mathcal{S}}
%% FANCY LETTERS %%
\newcommand{\fanc}{\mathscr{C}}
\newcommand{\fanb}{\mathscr{B}}
\newcommand{\fane}{\mathscr{E}}
\newcommand{\fanf}{\mathscr{F}}
\newcommand{\fanp}{\mathscr{P}}
\newcommand{\fanl}{\mathscr{L}}
\newcommand{\fana}{\mathscr{A}}
\newcommand{\fanx}{\mathscr{X}}
\newcommand{\fany}{\mathscr{Y}}
\newcommand{\fanz}{\mathscr{Z}}
\newcommand{\fang}{\mathscr{G}}
\newcommand{\fanh}{\mathscr{H}}
\newcommand{\fani}{\mathscr{I}}
\newcommand{\fanm}{\mathscr{M}}
\newcommand{\fann}{\mathscr{N}}
\newcommand{\fans}{\mathscr{S}}
%%%%%%%%%%%%%%%%%%%%%%%%%%%%%%%%%%%%%%%%%%%%%%%%%%%%%%%%%%%


%%%%%%%%%% OPERATIONS, CONNECTORS & SEPARATORS %%%%%%%%%%%%
%%% Module Things %%
\newcommand{\tensor}{\otimes}
\newcommand{\bigtensor}{\bigotimes}
\newcommand{\dsum}{\oplus}
\newcommand{\bigdsum}{\bigoplus}
%% Maps %%
\newcommand{\inj}{\hookrightarrow}
\newcommand{\surj}{\twoheadrightarrow}
\newcommand{\bij}{\overset{\sim}{\to}}
\newcommand{\dlim}{\varinjlim}
\newcommand{\inlim}{\varprojlim}  
%% Topology Things %%
\renewcommand{\bar}[1]{\overline{#1}}
\newcommand{\csum}{\#}
\DeclareMathOperator{\topint}{int}
\newcommand{\topwedge}{\vee}
\newcommand{\bigtopwedge}{\bigvee}
%% Left-Righting Things %%
\newcommand{\set}[1]{\left\{ #1 \right\}}
\newcommand{\brangle}[1]{\left\langle #1 \right\rangle}
\renewcommand{\brack}[1]{\left(   #1 \right)}
\newcommand{\abs}[1]{\left| #1 \right|}
\newcommand{\norm}[1]{\left\lVert #1 \right \rVert}
\newcommand{\floor}[1]{\left\lfloor #1 \right\rfloor}
\newcommand{\ceil}[1]{\left\lceil #1 \right\rceil}
\newcommand{\sqbrack}[1]{\left[ #1 \right]}
%%%%%%%%%%%%%%%%%%%%%%%%%%%%%%%%%%%%%%%%%%%%%%%%%%%%%%%%%


%%%%%%%%%%%%%%% CUSTOM SYMBOLS FOR FUNZIES %%%%%%%%%%%%%%%%%
\newcommand{\fish}{ %% For R.V. w/ Poisson dist %%
  \mathchoice
    {\mathrel{\ooalign{$\alpha$\cr\hidewidth \kern -0.13em $\cdot$\hidewidth}}}% \displaystyle
    {\mathrel{\ooalign{$\alpha$\cr\hidewidth \kern -0.13em$\cdot$\hidewidth}}}% \textstyle
    {\mathrel{\ooalign{$\scriptstyle\alpha$\cr\hidewidth\kern -0.13em $\scriptstyle\cdot$\hidewidth}}}% \scriptstyle
    {\mathrel{\ooalign{$\scriptscriptstyle\alpha$\cr\hidewidth \kern -0.13em $\scriptscriptstyle\cdot$\hidewidth}}}% \scriptscriptstyle
}

\newcommand{\notimplies}{
\mathrel{{\ooalign{\hidewidth$\not\phantom{=}$\hidewidth\cr$\implies$}}}}


%%%%%%%%%%%%%%%%%%%%%%%%%%%%%%%%%%%%%%%%%%%%%%%%%%%%%%%%%%%


%%%%%%%%%%%%%%%% MISC OPS&COMMANDS %%%%%%%%%%%%%%%%%
%% Categorical tingz %%
\DeclareMathOperator{\Set}{\underline{\mathsf{Set}}}
\DeclareMathOperator{\Ab}{\underline{\mathsf{Ab}}}
\DeclareMathOperator{\Grp}{\underline{\mathsf{Grp}}}
\DeclareMathOperator{\Top}{\underline{\mathsf{Top}}}
\DeclareMathOperator{\Diff}{\underline{\mathsf{Diff}}}
\DeclareMathOperator{\Rmod}{\underline{\mathsf{R-Mod}}}
\DeclareMathOperator{\Zmod}{\underline{Z-\mathsf{Mod}}}
\DeclareMathOperator{\Amod}{\underline{\mathsf{A-Mod}}}
\DeclareMathOperator{\Vect}{\underline{\mathsf{Vect}}}
\DeclareMathOperator{\Ring}{\underline{\mathsf{Ring}}}
\DeclareMathOperator{\CRing}{\underline{\mathsf{CRing}}}
\DeclareMathOperator{\Meas}{\underline{\mathsf{Meas}}}
\DeclareMathOperator{\Met}{\underline{\mathsf{Met}}}
\DeclareMathOperator{\Cat}{\underline{\mathsf{Cat}}}
\DeclareMathOperator{\Man}{\underline{\mathsf{Man}}}
\DeclareMathOperator{\Sh}{\underline{\mathsf{Sh}}}
\DeclareMathOperator{\Presh}{\underline{\mathsf{PreSh}}}
\DeclareMathOperator{\PreSh}{\underline{\mathsf{PreSh}}}
\DeclareMathOperator{\Open}{\underline{\mathsf{Open}}}
\DeclareMathOperator{\Et}{\underline{\mathsf{Et}}}
\DeclareMathOperator{\Hom}{Hom}
\DeclareMathOperator{\mor}{Mor}
\DeclareMathOperator{\Mor}{Mor}
\DeclareMathOperator{\ob}{Ob}
\DeclareMathOperator{\Ob}{Ob}
\DeclareMathOperator{\op}{\mathsf{op}}
\DeclareMathOperator{\coker}{coker}
\DeclareMathOperator{\coim}{coim}
%% Algebra Tingz %%
\DeclareMathOperator{\spn}{span}
\DeclareMathOperator{\ord}{ord}
\DeclareMathOperator{\lcm}{lcm}
\DeclareMathOperator{\aut}{Aut}
\DeclareMathOperator{\Aut}{Aut}
\DeclareMathOperator{\End}{End}
\DeclareMathOperator{\rank}{rank}
\DeclareMathOperator{\gal}{Gal}
\DeclareMathOperator{\Gal}{Gal}
\DeclareMathOperator{\sep}{sep}
\newcommand{\ideal}{\trianglelefteq}
\DeclareMathOperator{\rad}{rad}
\DeclareMathOperator{\nil}{nil}
\DeclareMathOperator{\ann}{ann}
\DeclareMathOperator{\ab}{ab}
\DeclareMathOperator{\Frac}{Frac}
\DeclareMathOperator{\trdeg}{trdeg}
\DeclareMathOperator{\orb}{Orb}
\DeclareMathOperator{\stab}{Stab}
\newcommand{\acts}{\curverightarrow}
\newcommand{\actson}{\curverightarrow}
\newcommand{\action}{\curverightarrow}
%% ANT Tings %%
\newcommand{\zzz}[1]{\zz/#1 \zz}
\DeclareMathOperator{\pic}{Pic}
\DeclareMathOperator{\cl}{Cl}
\DeclareMathOperator{\bigdiv}{Div}
\DeclareMathOperator{\smalldiv}{div}
\DeclareMathOperator{\trace}{Tr}
\DeclareMathOperator{\tr}{Tr}
\DeclareMathOperator{\disc}{disc}
%% Function Tingz %%
\DeclareMathOperator{\id}{Id}
\DeclareMathOperator{\sgn}{sgn}
\DeclareMathOperator{\image}{Im}
\newcommand{\im}[1]{\mathfrak{Im} \left( #1 \right)}
\newcommand{\re}[1]{\mathfrak{Re} \left( #1 \right)}
\newcommand{\inv}[1]{#1^{-1}}
%% Alg Geo Tingz %%
\newcommand{\homsheaf}{\mathcal{H}\text{om}}
\DeclareMathOperator{\spec}{Spec}
\DeclareMathOperator{\mspec}{Max-Spec}
\DeclareMathOperator{\supp}{Supp}
\DeclareMathOperator{\ass}{Ass}
\DeclareMathOperator{\weakass}{WeakAss}
\DeclareMathOperator{\proj}{Proj}
\DeclareMathOperator{\et}{et}
%% Matrix Tingz %%
\DeclareMathOperator{\GL}{GL}
\DeclareMathOperator{\PSL}{PSL}
\DeclareMathOperator{\SL}{SL}
\DeclareMathOperator{\SO}{SO}
\DeclareMathOperator{\Sp}{Sp}
\DeclareMathOperator{\SU}{SU}
\DeclareMathOperator{\mat}{Mat}
\DeclareMathOperator{\mot}{Mot}
\DeclareMathOperator{\rep}{Rep}
\DeclareMathOperator{\res}{Res}
%% Analysis Tingz %%
\DeclareMathOperator{\grad}{grad}
\DeclareMathOperator{\curl}{curl}
\newcommand{\dA}{\, \mathrm{d}A}
\newcommand{\dq}{\, \mathrm{d}q}
\newcommand{\dr}{\, \mathrm{d}r}
\newcommand{\ds}{\, \mathrm{d}s}
\newcommand{\dt}{\, \mathrm{d}t}
\newcommand{\du}{\, \mathrm{d}u}
\newcommand{\dv}{\, \mathrm{d}v}
\newcommand{\dV}{\, \mathrm{d}V}
\newcommand{\dx}{\, \mathrm{d}x}
\newcommand{\dy}{\, \mathrm{d}y}
\newcommand{\dz}{\, \mathrm{d}z}
\newcommand{\dtheta}{\, \mathrm{d}\theta}
\newcommand{\domega}{\, \mathrm{d}\omega}
\newcommand{\del}{\partial}
\DeclareMathOperator{\vol}{Vol}
%% Topology Tingz %%
\DeclareMathOperator{\rel}{rel}
\newcommand{\lift}[1]{\widetilde{#1}}
%% Logic Tingz %%
\newcommand{\semantics}[1]{[\![\textnormal{$ #1 $\/}]\!]}
\newcommand{\entails}{\Vdash}
%% Prob Tingz %%
\newcommand{\given}{\mid}
\DeclareMathOperator{\opt}{opt}
\DeclareMathOperator{\var}{var}
\DeclareMathOperator{\bin}{Bin}
\DeclareMathOperator{\geo}{Geo}
\DeclareMathOperator{\poi}{Poi}
\DeclareMathOperator{\ber}{Ber}

%%%%%%%%%%%%%%%%%%%%%%%%%%%%%%%%%%%%%%%%%%%%%%%%%%%%%%%%%%%

%%%%%%%%%%%%%%%%%%%%%%%%%%%%%%%%%%%%%%%%%%%%%%%%%%%%%%%%%%%%
%%%%%%%%%%%%%%%%%%%%%%%%%%%%%%%%%%%%%%%%%%%%%%%%%%%%%%%%%%%%



%%%%%%%%%%%%%%%%%%%%%%%%%%%%%%%%%%%%%%%%%%%%%%%%%%%%%%%%%%%%
%%%%%%%%%%%%%%%% ENVIRONMENTS AND PROOFS %%%%%%%%%%%%%%%%%%%
%%%%%%%%%%%%%%%%%%%%%%%%%%%%%%%%%%%%%%%%%%%%%%%%%%%%%%%%%%%%

%%%%%%%% DEFN ENV %%%%%%%%%%%%%
\declaretheoremstyle[
headfont= \bfseries \large \color{black}, headpunct={\\},
notefont= \bfseries \color{Red3}, notebraces={}{},
bodyfont=\normalfont\color{black},
postheadspace=0em,
spaceabove=0pt,
mdframed={
  skipabove=10pt,
  skipbelow=0pt,
  topline=false,
  rightline=false,
  bottomline=false,
  leftline=false,
  roundcorner=0pt,
  innerleftmargin=0pt,
  innerrightmargin=10pt}
]{defstyle}

\declaretheoremstyle[
headfont= \color{DodgerBlue3},
notefont= \color{DodgerBlue3}, notebraces={}{},
bodyfont=\normalfont \itshape \color{black},
postheadspace=0.5em,
spaceabove=0pt,
mdframed={
  skipabove=0pt,
  skipbelow=0pt,
  topline=false,
  rightline=false,
  bottomline=false,
  leftline=false,
  roundcorner=0pt,
  innerleftmargin=15pt,
  innerrightmargin=10pt}
]{exstyle}


%%%%%%%%%%%%%%%%%%%%%%%%%%%%%%%%%%%%%%%%


\usepackage{titlesec}
\titleformat{\section}[display]{\Large}{\bfseries Section \thesection}{0.1em}{\bfseries \Huge \color{black}}[]
\titleformat{\subsection}[block]{\large \centering}{\S \thesubsection}{1em}{}[]

\usepackage{etoolbox}

\renewcommand{\thmtformatoptarg}[1]{ -- #1 \vspace{10pt}}

\declaretheorem[name=\textsc{},style=defstyle]{definition}
\declaretheorem[name=\textsc{Example}, style=exstyle, parent=definition]{example}

%%%%%%%%%%%%%%%%%%%%%%%%%%%%%%%%%%%%%%%%%%%%%%%%%%%%%%%%%%%%
%%%%%%%%%%%%%%%%%%%%%%%%%%%%%%%%%%%%%%%%%%%%%%%%%%%%%%%%%%%%

\usepackage[titles]{tocloft}

\renewcommand{\cftsecfont}{\color{blue}}
\renewcommand{\cftsecpagefont}{}
\renewcommand{\cftsubsecfont}{\color{blue}}
\setlength\cftparskip{-10pt}
\setlength\cftbeforesubsecskip{10pt}

\begin{document}

\title{A Mathematical Glossary}

\author{Aaryan Sukhadia}
\date{Last Updated September 2024}

\maketitle

This is a list of words, phrases and sayings that are used by English-speaking higher mathematicians in writing or in speech. As cultures evolve and languages shift I expect this list to be fluid and ever-growing.\\

The goal of this project is to give a slight taster for the vernacular one might encounter in the jungle of mathematics, with the aim of providing people new to higher math a larger (and ideally more colorful) dictionary to work with. Even if you've read and/or written a lot of math already, I hope there's some terms in here that you might have never heard of, and end up sprinkling in your next mathematical conversation or exposition.\\

\textbf{What you WON'T find:}
\begin{itemize}
    \item Definitions of rigorous mathematical terms. This is better reserved for textbooks.
    \item Words or phrases that are commonly used in both math and general English texts, whose meanings don't differ in math. This includes words like "standard", "general" and "intuitive", for example. This does NOT include archaic English words, such as "whence", which is included in this glossary.
\end{itemize} 


\listoftheorems[numwidth=1em, title=Index, ignoreall, show={definition}]


\newpage

\begin{definition}[Abstract Nonsense]
\textbf{\textit{noun}}\\
Used (light-heartedly) to refer to long, theoretical parts of a mathematical argument that use a lot of very abstract concepts. Used very frequently for category theory or homology arguments.
\end{definition}

\begin{example}
It is a standard exercise in abstract nonsense to show left-adjoint functors commute with inverse limits.
\end{example}

\begin{definition}[Abuse of Notation]
\textbf{\textit{noun}}\\
The practice of using technically incorrect or non-rigorous but convenient shortcuts in written mathematics. Also often used: \textbf{Abuse of Language}.
\end{definition}

\begin{example}
With a slight \textbf{abuse of notation}, we write $f^n(x)$ for the $n$-th derivative of $f$.
\end{example}

\begin{example}
Let $X$ be a set and $\wp(X)$ denote the power set of $X$. Then an outer measure on $\wp(X)$ (or on $X$, by \textbf{abuse of language}), is defined as... 
\end{example}

\begin{definition}[Ansatz]
    
\end{definition}

\begin{definition}[Canonical]
\textbf{\textit{adjective}}\\
A mathematical object that is conventional, or the most "standard" or "reasonable" form of that object to use. Can also be used to refer to mathematical ideas in general, like proofs, concepts or techniques.
\end{definition}

\begin{example}
    There are uncountably many bases of $\rr^2$, but we usually work with the \textbf{canonical} basis $\set{[1, 0], [0, 1]}$.
\end{example}

\begin{example}
    Euclid's argument is the \textbf{canonical} proof of the infinitude of primes.
\end{example}

\begin{example}
    The \textbf{canonical} definition of a prime number is one with no proper, non-trivial factors.
\end{example}

\begin{definition}[Characterize]
    
\end{definition}

\begin{definition}[Chasing]
    
\end{definition}

\begin{definition}[De Facto]
    \textbf{\textit{adjective}}\\
    A Latin phrase meaning 
    
\end{definition}

\begin{definition}[Deep]
    
\end{definition}

\begin{definition}[Degenerate]
\textbf{\textit{adjective}}\\
Refers to subclass of mathematical objects that are much simpler than other objects of the same type. In some fields (like graph theory), "degenerate" also has a more rigorous, specific definition.
\end{definition}

\begin{example}
    Note any three points in the plane form a triangle, but we ignore the degenerate case of the three points all being on one line (because then the triangle itself is simply a line segment).
\end{example}



\begin{definition}[Elegant]
    
\end{definition}


\begin{definition}[Elementary]
\end{definition}


\begin{definition}[Epsilon]
    
\end{definition}

\begin{definition}[Folklore]
    \textbf{\textit{adjective}}\\
    Refers to a result that is well-known and accepted by most experts in a field, but for which there are little to no examples of a fully-written proof.
\end{definition}

\begin{example}
    Most introductory topology classes will introduce the concept of a connect-sum of connected surfaces, and will impicitly assume it is well-defined independent of choice of neighborhood. However since the rigorous proof of this is highly complex, many texts exclude it entirely, making it a \textbf{folklore} result.
\end{example}


\begin{definition}[Hand-Wave]
    
\end{definition}

\begin{definition}[Identity]
    
\end{definition}

\begin{definition}[Iff]
\textbf{\textit{abbreviation}}\\
Shorthand for "if and only if" (recall "A if and only if B" means "A implies B AND B implies A" i.e A and B are mathematically equivalent statements). Often used in the statement of definitions.
\end{definition}

\begin{example}
We say a function $f: X \to Y$ between topological spaces is continuous \textbf{iff} the pre-image of any open set in $Y$ is an open set in $X$.
\end{example}

\begin{example}
Recall that a real-valued matrix is invertible \textbf{iff} it has nonzero determinant.
\end{example}


\begin{definition}[Immediate]
\textbf{\textit{adjective}}\\
Something that is a simple consequence of, or follows easily from, a previously stated result.
\end{definition}
\begin{example}
    Consider the function $f(x) = \sqrt{x^2 + y^2}$. Continuity of this function is immediate, and...
\end{example}


\begin{definition}[Inspection]
    
\end{definition}

\begin{definition}[Ipso Facto]
    
\end{definition}

\begin{definition}[Modulo]
    
\end{definition}

\begin{definition}[Morally]
\textbf{\textit{adverb}}\\
An object or concept is "morally X" if one should expect it to be "X" 
    
\end{definition}

\begin{definition}[QED]
    
\end{definition}

\begin{definition}[Pathological]
\textbf{\textit{adjective}}\\
Referring to an example or object that exhibits a lot of unintuitive or unexpected properties. (Noun: \textbf{pathology}).
\end{definition}
\begin{example}
    The rational comb is the standard example of a \textbf{pathological} space that is connected but not path-connected.
\end{example}

\begin{example}
    Real analysis is full of \textbf{pathologies} like the Weierstrass function.
\end{example}

\begin{definition}[Serious Fact]
A mathematical result that has important consequences and is not simple to prove. Used liberally by Brian Conrad.
\end{definition}

\begin{example}
It's a serious fact 
\end{example}


\begin{definition}[TFAE]
    
\end{definition}

\begin{definition}[Toy (Example/Theorem/Problem)]

\end{definition}

\begin{definition}[Up to]
    
\end{definition}

\begin{definition}[Vacuously]
    
\end{definition}

\begin{definition}[Visibly]
    
\end{definition}

\begin{definition}[WLOG]
    \textbf{\textit{abbreviation}}\\
    An acronym for Without Loss Of Generality. Used before making an arbitrary choice that doesn't affect the overall mathematical argument. Can also be used all in lower case.
\end{definition}
\begin{example}
    Take real numbers $x, y$ and \textbf{wlog} let $x \leq y$.
\end{example}

\begin{example}
    Suppose $f$ is a constant function. \textbf{WLOG} let $f(x) = 1$ for all $x$.
\end{example}

 

\begin{definition}[Well-behaved]
    
\end{definition}

\begin{definition}[Whence]
    
\end{definition}



\end{document}